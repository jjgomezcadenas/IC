\subsection{Training}
\noindent The current idea is to choose random points over an $N \times N$ grid of SiPM planes and generate SiPM responses for all sensors in the plane using the S2 parameterization.  Note that $N$ must be chosen to minimize the number of inputs while still capturing the relevant amount of SiPM response.  Knowing the true location from which the light was generated, the net can be trained over many such events.  Since the energy correction table will attempt to provide a correction factor for each $(x,y)$ location, the DNN will be constructed to yield a single $(x,y)$ pair for an input SiPM map.  

\subsubsection{Questions}
\begin{enumerate}
	\item[-] Do we include SiPM noise in the training events?
	\item[-] Do we use ``toy'' events generated using the SiPM parameterization or will we use full-MC Kr events?
\end{enumerate}

\subsection{Application}
\noindent The $N \times N$ subset of SiPMs containing the responses generated by the EL light produced by a real (or full-MC) Kr x-ray must be isolated from all of the sensors in the tracking plane.  We should attempt to centralize the point (the hottest SiPM should be near the center of the $N \times N$ subsection) unless the event occurred near the edge of the tracking plane, making this impossible.  Either way, the DNN should be prepared to properly reconstruct non-centralized points.